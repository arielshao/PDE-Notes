%%%%%%%%%%%%%%%%%%%%%%%%%%%%%%%%%%%%%%%%%
% Short Sectioned Assignment
% LaTeX Template
% Version 1.0 (5/5/12)
%
% This template has been downloaded from:
% http://www.LaTeXTemplates.com
%
% Original author:
% Frits Wenneker (http://www.howtotex.com)
%
% License:
% CC BY-NC-SA 3.0 (http://creativecommons.org/licenses/by-nc-sa/3.0/)
%
%%%%%%%%%%%%%%%%%%%%%%%%%%%%%%%%%%%%%%%%%

%----------------------------------------------------------------------------------------
%	PACKAGES AND OTHER DOCUMENT CONFIGURATIONS
%----------------------------------------------------------------------------------------

\documentclass[paper=a4, fontsize=11pt]{scrartcl} % A4 paper and 11pt font size
\usepackage{graphicx}
\usepackage{wrapfig}
\usepackage[T1]{fontenc} % Use 8-bit encoding that has 256 glyphs
%\usepackage{fourier} % Use the Adobe Utopia font for the document - comment this line to return to the LaTeX default
\usepackage[english]{babel} % English language/hyphenation
\usepackage{amsmath,amsfonts,amsthm} % Math packages
\usepackage{sectsty} % Allows customizing section commands
\usepackage{changes}
\usepackage{amsmath,mathtools}
\usepackage{cancel}
\setdeletedmarkup{\cancel{#1}}


\allsectionsfont{\centering \normalfont\scshape} % Make all sections centered, the default font and small caps

\usepackage{fancyhdr} % Custom headers and footers
\pagestyle{fancyplain} % Makes all pages in the document conform to the custom headers and footers
\fancyhead{} % No page header - if you want one, create it in the same way as the footers below
\fancyfoot[L]{} % Empty left footer
\fancyfoot[C]{} % Empty center footer
\fancyfoot[R]{\thepage} % Page numbering for right footer
\renewcommand{\headrulewidth}{0pt} % Remove header underlines
\renewcommand{\footrulewidth}{0pt} % Remove footer underlines
\setlength{\headheight}{13.6pt} % Customize the height of the header

\numberwithin{equation}{section} % Number equations within sections (i.e. 1.1, 1.2, 2.1, 2.2 instead of 1, 2, 3, 4)
\numberwithin{figure}{section} % Number figures within sections (i.e. 1.1, 1.2, 2.1, 2.2 instead of 1, 2, 3, 4)
\numberwithin{table}{section} % Number tables within sections (i.e. 1.1, 1.2, 2.1, 2.2 instead of 1, 2, 3, 4)
\newtheorem{theorem}{Theorem}
 \newtheorem{definition}{Definition}
 \newtheorem{example}{Example}
  \newtheorem{exercise}{Exercise}
  \newtheorem{remark}{Remark} 
   \newtheorem{proposition}{Proposition}
   \newtheorem{lemma}{Lemma}
  \numberwithin{exercise}{section}
  
\setlength\parindent{0pt} % Removes all indentation from paragraphs - comment this line for an assignment with lots of text

%----------------------------------------------------------------------------------------
%	TITLE SECTION
%----------------------------------------------------------------------------------------

\newcommand{\horrule}[1]{\rule{\linewidth}{#1}} % Create horizontal rule command with 1 argument of height

\title{	
\normalfont \normalsize 
%
\horrule{0.5pt} \\[0.4cm] % Thin top horizontal rule
\huge Paper Review:  Regularity of Minimizers of Semilinear Elliptic Problems Up to Dimension 4 by Xavier Cabr\'{e}  \\ % The assignment title
\horrule{2pt} \\[0.5cm] % Thick bottom horizontal rule
}
\date{}
\author{Aili Shao} 


\begin{document}

\maketitle % Print the title

%----------------------------------------------------------------------------------------
%	PROBLEM 1
%----------------------------------------------------------------------------------------

\section{ Introduction}

Let $f\colon\mathbb{R}\to\mathbb{R}$ be a $C^{\infty}$ function and $F$ a primitive of $f$, i.e. $F'=f$. Let $\Omega\subset \mathbb{R}^n$ be a bounded, smooth domain.

Consider the semilinear PDE 
\begin{equation}\label{pde}       
\begin{cases} -\Delta u=f(u)\:\: \mbox{ in } \Omega,\\
u=0  \:\:\:\:\:\: \:\:\:\:\:\:\:\mbox{  on  }      \partial \Omega.

\end{cases}
\end{equation}
Its energy functional is 
\begin{equation}\label{fun}
E(u)=\int_{\Omega} \frac{1}{2} |\nabla u|^2-F(u)dx.
\end{equation}
\begin{itemize}


\item A function $u\in C_{0}^1(\bar{\Omega})$ is a local minimizer of (\ref{fun}) if there exists $\varepsilon>0$ such that 
$$E(u)\leq E(u+\xi)$$
for every $\xi\in  C_{0}^1(\bar{\Omega})$ such that $\|\xi\|_{C_1(\bar{\Omega})}\leq \varepsilon.$
\item A classical solution $u\in C^2(\bar{\Omega})$ of (\ref{pde}) is \emph{semi-stable} if
\begin{equation}\label{stable}
Q_{u}[\xi]:=\int_{\Omega} |\nabla \xi|^2-f'(u)\xi^2 dx\geq 0 \: \:\forall  \:\xi \in C_{0}^1(\bar{\Omega})
\end{equation}
\end{itemize}
\begin{remark}
\begin{itemize}
\item By elliptic regularity, every local minimizer $u$ is a $C^\infty$ classical solution to ( \ref{pde}).
\item The semistability of a solution $u$ is equivalent to the condition $\lambda_1\geq 0$ where $\lambda_1$ is the first Dirichlet eigenvalue of the linearized operator $-\Delta -f'(u)$  as
$$\lambda_1=\frac{\int_{\Omega}|\nabla u|^2-f'(u) u^2 dx}{\int_{\Omega} u^2}.$$
\item A local minimizer is always semi-stable. Note that is $u$ is a minimizer, then 
$$\frac{d^2}{dt^2}\Big |_{t=0} E[u+tv]\geq 0.$$

$$\frac{d}{dt} E[u+tv]=\int_{\Omega} (\nabla u +t\nabla v) \nabla v)- f(u+tv)v$$
$$\frac{d^2}{dt^2} E[u+tv]=\int_{\Omega} |\nabla v|^2-f'(u+tv)v^2$$

$$\frac{d^2}{dt^2}\Big |_{t=0} E[u+tv]=\int_{\Omega} |\nabla v|^2-f'(u)v^2\geq 0.$$
\end{itemize}

\end{remark}
\section{Main Estimate and Proof}
\begin{theorem}
Let $f$ be any $C^\infty$ function and $\Omega\subset \mathbb{R}^n$ be any smooth and bounded domain. Assume $2\leq n\leq 4.$ Let $u\in C_{0}^1(\bar{\Omega})$ with $u>0$ in $\Omega$ be a local minimizer of (
\ref{fun}) or more generally a positive classical semi-stable solution of (\ref{pde}). Then for every $t>0$,
\begin{equation}\label{main}
\|u\|_{L^\infty(\Omega)}\leq t+\frac{C}{t}|\Omega|^{\frac{4-n}{2n}} \left(\int_{\{u<t\}} |\nabla u|^4 dx \right) ^{\frac{1}{2}},
 \end{equation}
 where $C$ is a universal constant (in particular, independent of $f,\Omega$ and $u$).
 \end{theorem}
 
 Key Ingredients Needed to Prove Theorem 1.
 \begin{itemize}
 \item  \emph{Sard Lemma} (proved in Aili's 3rd Year Extended Essay)
 
 Let $\Omega$ be an open subset of $\mathbb{R}^n$ and $u\colon \Omega\to\mathbb{R}^n$ be $C^1(\Omega)$, then the measure of  the set of critical of $u$ values is zero. In particular, the set of regular values of $u$ is dense in $\mathbb{R}^n.$
 
 
 \item \emph{Coarea Formula} (Proof involves Fubini Theorem)
 
 Let $\Omega$ be an open set in $\mathbb{R}^n$. and $ u$ is a real-valued Lipschitz function on $\Omega$. Then for $g\in L^1(\Omega)$, 
 $$\int_{\Omega} g(x) |\nabla u(x)| dx=\int_{0}^T \int_{\Gamma_s} g(x) dV_{s} ds$$
 where $T:=\max_{\Omega} u=\|u\|_{L^{\infty}(\Omega)}$, $\Gamma_s:=\{x\in\Omega \colon u(x)=s\}.$
 
 \item \emph{Simon Michael Sobolev Inequality}
 
 Let $M\subset \mathbb{R}^{m+1}$ be $C^{\infty}$ immersed, $m$-dimensional compact hypersurface without boundary. Then for every $p\in [1,m)$, there exists a constant $C=C(m,p)$ such that for every $C^\infty$ function $v\colon M\to\mathbb{R}$,
 \begin{equation}\label{sobolev}
 \left(\int_{M} |v|^{p^\star} dV\right) ^{\frac{1}{p^{\star}}}\leq C(m,p) \left(\int_{M}|\nabla v|^p+|Hv|^p dV\right) ^{\frac{1}{p}} , 
 \end{equation}
 
 where $H$ is the mean curvature of $M$ and $p^\star=\frac{mp}{m-p}.$
 \item  \emph{Curvature Inequality}

$|H|\leq |A|$ where $H$ is the mean curvature of a surface $M$ defined as  $H:=\frac{1}{n-1}\sum_{1}^{n-1} k_{l}$ while $A:=(\sum_{1}^{n-1} k_{l}^2)^{1/2}$
 \item \emph{Sternberg and Zumbrun Inequality} (proved in class)
 
 Let $\Omega\subset \mathbb{R}^n$ be a smooth, bounded domain and $u$ is a smooth, positive , semi-stable solution of (\ref{pde}). Then for every Lipschitz continuous function $\eta$ in $\bar{\Omega}$ with $\eta\mid_{\partial\Omega}=0,$
\begin{equation} \label{SZ}
\int_{\Omega\cap \{| \nabla u|>0\} } \left(|\nabla_T | \nabla u| |^2+|A|^2 |\nabla u|^2\right)\eta^2 dx \leq \int_{\Omega} | \nabla u|^2 | \nabla \eta |^2 dx,
 \end{equation}
 where $\nabla_T$ denotes the tangential or Riemannian gradient along a level set of $u$ and $A$ is defined as above.
 \begin{proof}
 Note that 

$$ Q_{u}=\int_{\Omega} |\nabla \xi|
^2 -f'(u)\xi^2 dx\geq 0 $$
holds for every Lipschitz function $\xi$ in $\bar{\Omega}$ with $\xi\mid_{\partial \Omega}=0$ as $C_{0}^1(\bar{\Omega})$ is dense in this space. Take $\xi=c\eta$ in the above inequality where $c$ is a smooth function while $\eta$ is Lipschitz continuous in $\bar{\Omega}$ and $\eta\mid_{\partial \Omega}=0.$
\begin{align*}
Q_{u}[c\eta]=&\int_{\Omega} |\nabla (c\cdot\eta) |^2-f'(u) c^2 \eta^2 dx\\
{}=&\int_{\Omega} |\nabla c\cdot\eta +c\cdot \nabla \eta|^2-f'(u) c^2 \eta^2 dx\\
{}=&\int_{\Omega} c^2|\nabla \eta|^2 +2 \int_{\Omega} \nabla c\cdot c \nabla \eta \cdot \eta +\int_{\Omega}\eta^2 |\nabla c|^2- \int_{\Omega} f'(u) c^2\eta^2 dx\\
{}=&\int_{\Omega}  c^2  | \nabla \eta|^2 +\deleted{ \int_{\Omega}\nabla(\eta^2\nabla c\cdot c) }-\int_{\Omega} \eta^2 \Delta c\cdot c-\int_{\Omega} f'(u) c^2 \eta^2 dx\\
{}=&\int_{\Omega} c^2 |\nabla \eta|^2-(\Delta c+f'(u) c)c \eta^2 dx.\\
\end{align*}
Thus the semi-stability condition gives
\begin{equation}\label{semistability}
Q_{u}[c\eta]=\int_{\Omega} c^2 |\nabla \eta|^2-(\Delta c+f'(u) c)c \eta^2 dx\geq 0.
\end{equation}
Take $c=\sqrt{|\nabla u|^2 +\varepsilon^2}$ for a give $\varepsilon>0.$ $c$ is smooth.

$$\Delta u+f(u)=0 \mbox{ in } \Omega.$$
$$\Delta u_j+f'(u)u_j=0 \mbox{ in } \Omega.$$
$$c_j=\frac{1}{2}\frac{1}{\sqrt{|\nabla u|^2 +\varepsilon^2}} \cdot 2 \sum_{i=1}^n u_i u_{ij}=\frac{1}{\sqrt{|\nabla u|^2 +\varepsilon^2}} \cdot\sum_{i=1}^n u_i u_{ij}.$$
\begin{align*}
c_{j}j=&\frac{1}{\sqrt{|\nabla u|^2 +\varepsilon^2}} \cdot\sum_{i=1}^n  u_{ij}^2.+\frac{1}{\sqrt{|\nabla u|^2 +\varepsilon^2}} \cdot\sum_{i=1}^n u_i u_{ijj}\\
{}+&\frac{1}{(\sqrt{|\nabla u|^2 +\varepsilon^2})^3} \cdot(-\frac{1}{2})\sum_{i=1}^n 2u_i u_{ij} \cdot \sum_{i=1}^n u_i u_{ij}
\end{align*}

\begin{align*}
\sum_{j=1}^n c_{jj}=& \frac{1}{|\nabla u|^2 +\varepsilon^2}\left[ \sum_{i,j}^n u_{ij}^2 \sqrt{|\nabla u|^2 +\varepsilon^2} +\sum_{i=1}^n u_i \Delta u_i \sqrt{|\nabla u|^2 +\varepsilon^2}\right] \\
{}& -\frac{1}{|\nabla u|^2 +\varepsilon^2}\left[ \sum_{j=1}^n (\sum_{i=1}^n u_{ij} u_i)^2 )\frac{1}{\sqrt{|\nabla u|^2 +\varepsilon^2}} \right]
\end{align*}

That is,
\begin{align*}
\Delta c=&\frac{1}{|\nabla u|^2 +\varepsilon^2} \left[-f'(u)|\nabla u|^2 \sqrt{|\nabla u|^2 +\varepsilon^2}+\sum_{i,j}^n u_{ij}^2\sqrt{|\nabla u|^2 +\varepsilon^2} \right]\\
{}-&\frac{1}{|\nabla u|^2 +\varepsilon^2} \left[ (\sum_{i=1}^n u_{ij} u_i)^2 )\frac{1}{\sqrt{|\nabla u|^2 +\varepsilon^2}} \right].
\end{align*}
Thus, 
\begin{align*}
\Delta c+f'(u) c=& f'(u) \sqrt{|\nabla u|^2 +\varepsilon^2}-\frac{f'(u)|\nabla u|^2}{sqrt{|\nabla u|^2 +\varepsilon^2}} \\
{}+& \frac{1}{\sqrt{|\nabla u|^2 +\varepsilon^2}}\left[\sum_{i,j}^n u_{ij}^2-\sum_{j}^n\left(\sum_{i}^n u_{ij} \frac{u_i}{\sqrt{|\nabla u|^2 +\varepsilon^2}}\right)^2 \right]\\
{}=& f'(u)\frac{\varepsilon^2}{\sqrt{|\nabla u|^2 +\varepsilon^2}}+\frac{1}{\sqrt{|\nabla u|^2 +\varepsilon^2}} \left[\sum_{ij} ^nu_{ij}^2-\sum_{j}^n\left(\sum_{i}^nu_{ij} \frac{u_i}{\sqrt{|\nabla u|^2 +\varepsilon^2}} \right)^2 \right]\\
\end{align*}
Using the semi-stable inequality (\ref{semistability}), we deduce 
\begin{align*}
\int_{\Omega}(|\nabla u|^2 +\varepsilon^2)|\nabla \eta|^2dx =& \int_{\Omega} c^2 |\nabla \eta|^2 dx\\
{}\geq & \int_{\Omega} (\Delta c+f'(u) c) c\eta^2) dx\\
{}=&\int_{\Omega}f'(u)\varepsilon^2 \eta^2 dx+\int_{\Omega}\left[\sum_{ij}^n u_{ij}^2-\sum_{j}^n\left(\sum_{i}^nu_{ij} \frac{u_i}{\sqrt{|\nabla u|^2 +\varepsilon^2}} \right)^2\right]\eta^2 dx 
\end{align*}
The integrand in the last integral is non-negative, so we have 
\begin{align*}
&\int_{\Omega}\left[\sum_{ij}^n u_{ij}^2-\sum_{j}^n\left(\sum_{i}^nu_{ij} \frac{u_i}{\sqrt{|\nabla u|^2 +\varepsilon^2}} \right)^2\right]\eta^2 dx  \\
\geq &\int_{\Omega\cap \{|\nabla u|>0\}}\left[\sum_{ij}^n u_{ij}^2-\sum_{j}^n\left(\sum_{i}^nu_{ij} \frac{u_i}{\sqrt{|\nabla u|^2 +\varepsilon^2}} \right)^2\right]\eta^2 dx \\
\geq &\int_{\Omega\cap \{|\nabla u|>0\}}\left[\sum_{ij}^n u_{ij}^2-\sum_{j}^n\left(\sum_{i}^nu_{ij} \frac{u_i}{|\nabla u|} \right)^2\right]\eta^2 dx 
\end{align*}

Thus, 
\begin{align*}
\int_{\Omega}(|\nabla u|^2 +\varepsilon^2)|\nabla \eta|^2dx \geq
&\int_{\Omega}f'(u)\varepsilon^2 \eta^2 dx+\int_{\Omega\cap \{|\nabla u|>0\}}\left[\sum_{ij}^n u_{ij}^2-\sum_{j}^n\left(\sum_{i}^nu_{ij} \frac{u_i}{|\nabla u|} \right)^2\right]\eta^2 dx 
\end{align*}
Letting $\varepsilon\to 0$, we have 
\begin{align*}
\int_{\Omega}|\nabla u|^2|\nabla \eta|^2dx \geq
&\int_{\Omega\cap \{|\nabla u|>0\}}\left[\sum_{ij}^n u_{ij}^2-\sum_{j}^n\left(\sum_{i}^nu_{ij} \frac{u_i}{|\nabla u|} \right)^2\right]\eta^2 dx 
\end{align*}
It remains to show that 
\begin{equation}\label{Aequality}
\sum_{ij}^n u_{ij}^2-|\nabla |\nabla u||^2 =|\nabla_{T}|\nabla u||^2+|A|^2|\nabla u|^2.
\end{equation}
Proof of (\ref{Aequality}):
Fix $x_0$ such that $|\nabla u(x_0)|\neq 0.$ Define $\tau_n:=\frac{\nabla u(x_0)}{|\nabla u(x_0)|}$ be the normal direction and $\tau_1, \tau_2, \cdots, \tau_{n-1}$ be the tangential directions. 

\begin{align*}
\nabla u_j =& (\nabla u_j \cdot \tau_n) \tau_n+\sum_{i=1}^{n-1} (\nabla u_j \cdot \tau_i)\tau_i \\
{}=& (\nabla u_j \cdot \frac{\nabla u}{|\nabla u|})\tau_n+\sum_{i=1}^{n-1} (\nabla u_j \cdot \tau_i)\tau_i \\
{}=& \frac{1}{2} (|\nabla u|)_j\frac{1}{|\nabla u|} \tau_n +\sum_{i=1}^{n-1} (\nabla u_j \cdot \tau_i)\tau_i \\
{}=& (|\nabla u|)_j \tau_n +\sum_{i=1}^{n-1} (\nabla u_j \cdot \tau_i)\tau_i \\
\end{align*}
Thus $ |\nabla u_j|^2 =(|\nabla u |)_j^2 +\sum_{i=1}^{n-1} (\nabla u_j \cdot \tau_i)^2$.
This implies that 
$$\sum_{i,j=1}^n u_{ij}^2 =\sum_{j=1}^n|\nabla u_j|^2=|\nabla |\nabla u||^2 +\sum_{j=1}^n\sum_{i=1}^{n-1} (\nabla u_j \cdot \tau_i)^2.$$
Now it remains to show that 
$$\sum_{j=1}^n\sum_{i=1}^{n-1} (\nabla u_j \cdot \tau_i)^2=|A|^2|\nabla u|^2 +|\nabla_{T} |\nabla u||^2.$$
Note that since $\nabla u =|\nabla u| \tau_n$, we have 
$$\nabla u_j \tau_i=\frac{\partial}{\partial x_j} (|\nabla u| \tau_n)\tau_i =|\nabla u|\tau_{n.j} \tau_i$$
where $\tau_{n,j}=\frac{\partial}{\partial x_j} (\tau_n).$
Therefore, 
\begin{align*}
\sum_{j=1}^n\sum_{i=1}^{n-1} (\nabla u_j \cdot \tau_i)^2=& |\nabla u|^2 \sum_{j=1}^n \sum_{i=1}^{n-1} (\tau_{n,j} \tau_i)^2\\
{}=& |\nabla u|^2 \sum_{j=1}^{n-1} \sum_{i=1}^{n-1} (\tau_{n,j} \tau_i)^2+|\nabla u|^2 \sum_{i=1}^{n-1} (\tau_{n,n} \tau_i)^2\\
{}=&|\nabla u|^2 \sum_{i,j=1}^(n-1) h_{ij}^2 +|\nabla_{T} |\nabla u||^2\\
{}=&|A|^2|\nabla u|^2 +|\nabla_{T} |\nabla u||^2.
\end{align*}
 \end{proof}
 \item \emph{Geometric Inequality for $\Gamma_s$}
 
$$|\Gamma_s|^{\frac{n-2}{n-1}}\leq C(n) \int_{\Gamma_s} |H| dV_s$$
It  follows from the \emph{Simon Michael Sobolev Inequality} by taking $v\equiv 1$, $m=n-1>1=p$, $M=\Gamma_s$. This inequality also holds if $\Gamma_s$ is not connected.

\item \emph{Isoperimetric Inequality}
 $$V(s):=|\{u>s \}|\leq C(n) |\Gamma_s|^{\frac{n}{n-1}}$$
 \end{itemize}


Now we proceed to the proof of Theorem 1.

\begin{proof}
\begin{itemize}
\item  \emph{Step 1: Set Up}.

By elliptic regularity theory, $u\in C^{\infty }(\bar{\Omega})$. Recall that $u>0$ in $\Omega$. We define 
$$T:=\max_{\Omega} u =\|u\|_{L^\infty(\Omega)}.$$

For $s\in (0,T), \Gamma_s:=\{x\in\Omega \colon u(x) =s\}.$ 

By Sard's Lemma, almost every $s\in (0,T)$ is a regular value of $u$. By definition, $|\nabla u(x)|>0$ for all $x\in \Gamma_s.$ In particular, if $s$ is a regular value, $\Gamma_s$ is a $C^{\infty}$-immersed compact hypersurface of $\mathbb{R}^n$ without boundary. ( Later we will apply the Simon Michael Sobolev inequality with $M=\Gamma_s$. Note that $\Gamma_s$ could have a finite number of connected components, the Simon Michael Sobolev inequality still holds. )
\item  \emph{Step 2: Apply Semi-stability Condition and Sternberg  Zumbrun Inequality}.

Since $u$ is a semi-stable solution,  we  can apply the Sternberg Zumbrun inequality.
Take $$\eta (x) =\varphi (u(x)) \mbox { for } x\in\Omega $$
where $\varphi$ is a Lipschitz function on $[0,T]$ with $\varphi(0)=0.$

Now the RHS of the Sternberg  Zumbrun Inequality becomes 
\begin{align*}
\int_{\Omega} |\nabla u|^2 |\nabla \eta|^2 dx=& \int_{\Omega} |\nabla u|^4 \varphi'(u)^2 dx\\
{}=& \int_{0}^T \left( \int_{\Gamma_s} |\nabla u|^3 dV_s \right) \varphi'(s)^2 ds \mbox{ By Coarea Formula}\\
\end{align*}
The integral in $ds$ is over the regular values of $u$ whose complement is of zero measure in $(0,T).$

For the LHS of the inequality, we integrate ove $\Omega\cap\{|\nabla u|>\delta \}$ for a given $\delta >0$, then the inequality still holds. That is,\
\begin{align*}
\int_{0}^T \left( \int_{\Gamma_s} |\nabla u|^3 dV_s \right) \varphi'(s)^2 ds \geq & \int_{\Omega\cap\{|\nabla u|>\delta \}} \left( |\nabla_{T}|\nabla u||^2+|A|^2|\nabla u|^2 \right) \varphi(u)^2 dx\\
{}=&\int_{0}^T \left(\int_{\Gamma_s\cap\{|\nabla u|>\delta\}} \frac{1}{|\nabla u|} \left( |\nabla_{T}|\nabla u||^2+|A|^2|\nabla u|^2 \right) dV_s \right) \varphi(s)^2 ds\\
{}=&\int_{0}^T \left(\int_{\Gamma_s\cap\{|\nabla u|>\delta\} } 4\left( |\nabla_{T}|\nabla u|^{1/2}|^2+(|A||\nabla u|^{1/2})^2 \right) dV_s \right) \varphi(s)^2 ds\\
\end{align*}
Letting $\delta\to 0$, by Monotone Convergence Theorem, we have 
\begin{equation}\label{hequation}
\int_{0}^T h_1(s) \varphi(s)^2 ds \leq \int_{0}^T h_2(s) \varphi'(s)^2 ds
\end{equation}
for all Lipschitz functions $\varphi\colon [0,T]\to \mathbb{R}$ with $\varphi(0)=0$ 
where 
$$h_1(s):=\int_{\Gamma_s}4\left( |\nabla_{T}|\nabla u|^{1/2}|^2+(|A||\nabla u|^{1/2})^2 \right) dV_s $$
and 
$$h_2(s):=\int_{\Gamma_s} |\nabla u|^3 dV_s$$
for every regular value  $s$ of $u$.


\item \emph{Step 3: Apply the Sobolev Inequality}


In this step, we apply the Simon Michael Sobolev inequality to argue the reason for restricting $n\leq 4.$
Take $M=\Gamma_s, p=2 <m =n-1,  v=|\nabla u|^{1/2}$, then 
$$\left(\int_{\Gamma_s} |\nabla u|^{\frac{n-1}{n-3}} dV_s \right) ^{\frac{n-1}{n-3}}\leq C(n)\int_{\Gamma_s}|\nabla_{T}|\nabla u|^{1/2}|^2+(|H||\nabla u|^{1/2})^2 dV_s.$$
Then 
\begin{equation}\label{h1}
\left(\int_{\Gamma_s} |\nabla u|^{\frac{n-1}{n-3}} dV_s \right) ^{\frac{n-1}{n-3}}\leq c(n) h_1(s)
\end{equation}
as $|H|\leq |A|$.
Combining this with (\ref{hequation}) , we have 
\begin{equation}
\int_{0}^T \left(\int_{\Gamma_s} |\nabla u|^{\frac{n-1}{n-3}} dV_s \right) ^{\frac{n-1}{n-3}} \varphi(s)^2 ds \leq C(n) \int_{0}^T \left( \int_{\Gamma_s} |\nabla u|^3 dV_s \right) \varphi'(s)^2 ds
\end{equation}
for all Lipschitz functions $\varphi$ in $[0,T]$ with $\varphi(0)=0.$ So we need $\frac{n-1}{n-3}\geq 3.$ That is $n\leq 4.$

Now we define 
$$B_t:=\frac{1}{t^2}\int_{\{u<t\}} |\nabla u|^4 dx =\frac{1}{t^2}\int_{0}^t h_2(s) ds$$
where the last equality follows from the Coarea Formula.
\item \emph{Step 4: Proof for $n=4$}

When $n=4, \frac{n-1}{n-3}=3.$ Then (\ref{h1}) gives 
\begin{equation}\label{h1h2}
h_2^{1/3}\leq Ch_1  \mbox{ a.e.  in } (0,T) 
\end{equation}
where $C$ is a universal constant.
For every regular value $s$ of $u$, we have $0<h_2(s)$, $h_1(s)<\infty$, so $\frac{h_1}{h_2}\in (0,\infty)$ a.e.  in $(0,T)$.

$$g_k(s):=\min\{ k,\frac{h_1(s)}{h_2(s)} \}$$
for regular values $s$ and for a postive integer $k$, we have that $g_k\in L^\infty (0,T)$ and  $g_k(s) \to \frac{h_1(s)}{h_2(s)}\in (0, \infty) \mbox{ as } k\to
\infty.$ for a.e. $s\in(0,T).$
Since $g_k\in L^\infty (0,T)$,
$$\varphi_k(s):=\begin{cases} s/t   & \mbox{ if } s\leq t; \\
\exp\left(\frac{1}{\sqrt{2}}\int_{t}^s \sqrt{g_k(\tau)} d\tau\right)   &\mbox{ if } t\leq s\leq T\\
\end{cases}$$
is well defined and Lipschitz continuous in $[0,T]$ with $ \varphi_k(0)=0.$

Since $h_2(\varphi_{k}^{'})^2=h_2 \frac{1}{2} g_k \varphi_k^2\leq \frac{1}{2}h_1\varphi_k^2$ in $(t,T)$, by inequality (\ref{hequation}) ($\varphi=\varphi_k$), we have 
\begin{align*}
\int_{t}^T h_1 \varphi_k^2 ds \leq & \int_{t}^T h_2 (\varphi_{k}^{'})^2 ds+ \int_{0}^t h_2 (\varphi_{k}^{'})^2 ds\\
{}\leq &\frac{1}{2}\int_{t}^T h_1 \varphi_k^2 ds+\frac{1}{t^2} \int_{0}^t h_2 ds\\
\end{align*}
Thus, 
\begin{align*}
\int_{t}^T h_1 \varphi_k^2 ds \leq & \frac{2}{t^2}\int_{0}^t h_2 ds\\
{}=& \frac{2}{t^2}\int_{\{u<t\}} |\nabla u|^4 dx\\
{}=& 2B_t
\end{align*}
Note that we need to establish 
$$T-t\leq  C B_t^{1/2}.$$
\begin{align*}
T-t=& \int_{t}^T ds\\
{}=& \sup _{k\geq 1} \int_{t}^T \sqrt[4]{\frac{h_2}{h_1}g_k} ds\\
{}=&\int_{t}^T (\sqrt{h_1} \varphi_k) (\sqrt[4]{\frac{h_2g_k}{h_1^3}} \frac{1}{\varphi_k})ds\\
{}\leq& (2B_t)^{1/2}  \left[\int_{t}^T (\sqrt{\frac{h_2 g_k}{h_1^3} }\frac{1}{\varphi_k^2})ds\right]^{1/2}.\\
{}\leq & (2B_t)^{1/2} \left[ C\int_{t}^T \sqrt{g_k}\frac{1}{\varphi_k^2} ds \right]^{1/2}
\end{align*}
since  $h_2\leq Ch_1^3.$
Finally, we need to bound the integral on the RHS,
\begin{align*}
\int_{t}^T \sqrt{g_k}\frac{1}{\varphi_k^2} ds =& \int_{t}^T \sqrt{g_k}\frac{1}{\varphi_k^2}\frac{\varphi_{k}^{'}}{\frac{1}{\sqrt{2}}\sqrt{g_k}\varphi_k} ds\\
{}=& \sqrt{2} \int_{t}^T \frac{\varphi_{k}^{'}}{\varphi_{k}^3} ds\\
{}=&\frac{\sqrt{2}}{2} [\varphi_{k}^{-2}(s)]_{s=T}^{s=t} \\
{}\leq &\frac{\sqrt{2}}{2} \varphi_{k}^{-2}(t)\\
{}=&\frac{\sqrt{2}}{2}\\
\end{align*}
This implies that $ \int_{t}^T (\sqrt{h_1} \varphi_k) (\sqrt[4]{\frac{h_2g_k}{h_1^3}} \frac{1}{\varphi_k})ds\leq \sqrt{2}B_{t}^{1/2} \frac{\sqrt{2}}{2}=B_t^{1/2}.$
That is 
$$T-t\leq B_t^{1/2}.$$
Thus,
$$\|u\|_{L^{\infty}(\Omega)} \leq t +\frac{1}{t} \left(\int_{\{u<t\}} |\nabla u|^4 dx \right)^{1/2}.$$
This completes the proof for $n=4$.
\item \emph{Step 5: Proof for $n=2,3$}

Now we consider a simple test function 
$$\varphi(s)=\begin{cases} s/t & \mbox{ if } s\leq t;\\
1 & \mbox{ if } s>t.\\
\end{cases}$$
By definition of $h_1(s)$, we have $ h_1(s) \geq \int _{\Gamma_s} |A|^2|\nabla u| dV_s.$ Inequality (\ref{hequation}) leads to 
\begin{align*}
\int_{t}^T\int _{\Gamma_s} |A|^2|\nabla u| dV_s ds\leq & \int_{0}^T h_1(s) \varphi(s)^2 ds\\
{}\leq & \int_{0}^T h_2(s) (\varphi'(s))^2 ds\\
{}=&\frac{1}{t^2} \int_{0}^t h_2(s) ds\\
{}=& \frac{1}{t^2} \int_{\{u<t\}} |\nabla u|^4 dx\\
{}=& B_t
\end{align*}
This equality holds for ecery dimension $n$. It is at the end of the proof that we will need to assume $n\leq 3.$
Now we use the geometric inequality for $\Gamma_s$,
$$|\Gamma_s|^{\frac{n-2}{n-1}}\leq C(n) \int_{\Gamma_s} |H| dV_s$$
 and the \emph{isoperimetric inequality}
 $$V(s):=|\{u>s \}|\leq C(n) |\Gamma_s|^{\frac{n}{n-1}}$$
 to deduce an inequality about $V(s)$.
 \begin{align*}
 V(s)^{\frac{n-2}{n}}\leq & C(n) |\Gamma_s|^{\frac{n-2}{n-1}}\\
 {}\leq & C(n) \int_{\Gamma_s} |H| dV_s \\
 {}\leq & C(n)\left[ \int_{\Gamma_s} |A|^2|\nabla u| dV_s\right]^{1/2} \left[\int_{\Gamma_s} \frac{dV_s}{|\nabla u|}\right]^{1/2}
 \end{align*}
 for all regular values $s$ by Cauchy Schwarz and since $|H|\leq |A|.$
 Then 
 \begin{align*}
 T-t=&\int_{t}^T ds\\
 {}\leq & \int_{t}^T C(n)\left[ \int_{\Gamma_s} |A|^2|\nabla u| dV_s\right]^{1/2}\left[\int_{\Gamma_s} V(s)^{\frac{2(2-n)}{n}} \frac{dV_s}{|\nabla u|}\right]^{1/2}\\
 {}\leq & C(n) \left[  \int_{t}^T \int_{\Gamma_s} |A|^2|\nabla u| dV_s\right]^{1/2}\left[ \int_{t}^TV(s)^{\frac{2(2-n)}{n}} \int_{\Gamma_s} \frac{dV_s}{|\nabla u|}\right]^{1/2}\\
 {}\leq &C(n) B_t^{1/2}  \left[ \int_{t}^TV(s)^{\frac{2(2-n)}{n}} \int_{\Gamma_s} \frac{dV_s}{|\nabla u|}\right]^{1/2}\\
 \end{align*}
 Finally since $V(s)=|\{u>s\}|$ is non-increasing, it is differentiable a.e. by \emph{Coarea Formula}, we have
 $$-V(s)=\int_{\Gamma_s} \frac{dV_s}{|\nabla u|}  \mbox{ for a.e. } s\in (0,T).$$
 
 In addition, for $n\leq 3.$ $V(s)$ is non-increasing in $s$ and thus it is total variation 
 satisfies 
 \begin{align*}
 |\Omega|^{\frac{4-n}{n}} \geq & V(t)^{\frac{4-n}{n}}\\
 {}=& \left[ V(s) ^{\frac{4-n}{n}}\right ]_{s=T} ^{s=t} \\
 {}\geq & \int_{t}^T \frac{4-n}{n} V(s) ^{\frac{2(2-n)}{n}} (-V'(s)) ds\\
 {}=&\frac{4-n}{n} \int_{t}^T V(s)^{\frac{2(2-n)}{n}} \int_{\Gamma_s} \frac{dV}{|\nabla u|} ds
 \end{align*}
 Thus, 
 $$T-t\leq C(n) B_t^{1/2} |\Omega|^{\frac{4-n}{2n}} \mbox{ for } n\leq 3.$$
 Note that this argument gives nothing for $n\geq 4$ since the integral 
 $$\int_{t}^T V(s)^{\frac{2(2-n)}{n}} (-V'(s))ds=\int_{0}^{V(t)} \frac{dr}{r^{\frac{2(n-2)}{n}}},$$
 is not convergent at $s=T( r=0)$ because $\frac{2(n-2)}{n}\geq 1.$
\end{itemize}
\end{proof}
\section{Relevant  Results and Applications}
\begin{theorem}
Let $f$ be any $C^{\infty}$ function and $\Omega\subset \mathbb{R}^n$ any $C^{\infty}$ bounded domain. Assume that $2\leq n\leq 4$ and that $\Omega$ is convex in the case $n\in \{3,4\}$.
Let $u\in L^1(\Omega)$ be a positive weak solution of (\ref{pde}) and suppose that $u$ is the $L^1(\Omega)$ limit of a sequence of classical positive semistable solutions of (\ref{pde} ). We then have the following:
\begin{enumerate}
\item If $f\geq 0$ in $[0,\infty),$ then $u\in L^{\infty}(\Omega).$
\item Assume that $f(s)\geq c_1>0$ and $f(s)\geq \mu s-c_2$ for all $s\in[0,\infty)$ for some positive constants $c_1$ and $c_2$ and for $\mu >\lambda_1(\Omega)$, where $\lambda_1(\Omega)$ is the first Dirichlet eigenvalue of $-\Delta$ in $\Omega$. Then,
$$\|u\|_{L^{\infty}(\Omega)} \leq C(\Omega,\mu, c_1,c_2,\| f\|_{L^{\infty} ([0,\bar{C}(\Omega,\mu,c_2)]}),$$
where $C(\cdot)$ and $\bar{C}(\cdot)$ are constants depending only on the quantities within the parentheses.
\end{enumerate}
\end{theorem}
Before we proceed to the proof of Theorem 2, we prove the following propositions and lemma first:
\begin{proposition}
Let $f$ be any $C^\infty$ function. Let $\Omega\subset \mathbb{R}^n$ be any smooth bounded domain. Assume that $2\leq n\leq 4.$
Let $u$ be a classical semi-stable solution of (\ref{pde}). Assume that 
\begin{equation}\label{3.1}
u\geq c_3\mathrm{dist}(\cdot,\partial\Omega) \mbox{ in } \Omega
\end{equation}
and 
\begin{equation}\label{3.2}
\|u\|_{L^{\infty} (\Omega_{\varepsilon})}\leq c_4 \mbox{ where } \Omega_{\varepsilon}=\{x\in\Omega \colon :\mathrm{dist}(x,\partial \Omega)<\varepsilon\},
\end{equation}
for some positve constants $\varepsilon, c_3$ and $c_4$. Then,
\begin{equation}\label{3.3}
\|u\|_{L^{\infty}(\Omega)}\leq C(\Omega,\varepsilon,c_3,c_4,\|f\|_{L^{\infty}([0,c_4])}),
\end{equation}
where $C(\dot)$ is a constant depending only on the quantities within the parentheses.
\begin{proof}
By taking $\varepsilon $ smaller if necessary, we may assume that 
$$\Omega_{\delta}=\{x\in\Omega \colon \mathrm{dist}(x,\partial \Omega)<\delta \}$$ is smooth for every $0<\delta\leq \varepsilon.$ We use Theorem 1 with  the choice 
$$t=c_3\frac{\varepsilon}{2}.$$
Note that if $x\in \{ u<t\}$,  by (\ref{3.1}), we have 
$$c_3\mathrm{dist}(x,\partial \Omega) <t =c_3 \frac{\varepsilon}{2}.$$
This implies $$\mathrm{dist}(x,\partial\Omega) <\frac{\varepsilon}{2}.$$
Thus $\{u<t\}\subset \Omega_{\varepsilon/2}.$
Now it suffices to bound $\|u\|_{W^{1,4} \left(\Omega_{\varepsilon/2}\right)}.$

$u$ is a solution of 
$$\begin{cases} -\Delta u =f(u) &\mbox{ in } \Omega_{\varepsilon} \\
 u=0&  \mbox{ on } \partial\Omega.\\
 \end{cases}.$$
 By (\ref{3.2}), $\|u\|_{L^\infty(\Omega_{\varepsilon})} \leq c_4$ and thus the RHS of the PDE satisfies $$\|f(u)\|_{L^\infty (\Omega_{\varepsilon})} \leq \|f\|_{L^\infty( [0, c_4])}.$$
 Note that $f\in L^\infty(\Omega_{\varepsilon} )\subset L^4(\Omega_{\varepsilon/2}),$
 so by elliptic regularity, $\|u\|_{W^{2,4}(\Omega_{\varepsilon/2})}<\infty.$ We can thus conclude that $\|u\|_{W^{1,4}(\Omega_{\varepsilon/2})}$ is bounded.
 \end{proof}
\end{proposition}

Note that the $L^\infty$ bound in (\ref{3.2}) holds every Lipschitz nonlinear function $f$ when $\Omega$ is a convex domain (for $n\geq 2$). The precise statement is the following:
\begin{proposition}
Let $f$ be any locally Lipschitz function and let $\Omega\subset\mathbb{R}^n$ be a smooth bounded domain. Let $u$ be any positive classical solution of (\ref{pde}).

If $\Omega$ is convex, then there exist positive constants $\rho$ and $\gamma$ depending only on the domain $\Omega$ such that for every $\Omega$ with $\mathrm{dist}(x,\partial \Omega) <\rho$, there exists a set $I_x\subset \Omega$ with the following properties:
\begin{equation}\label{3.4}
|I_x|\geq \gamma \mbox{ and }  u(x) \leq u(y)  \mbox{ for all } y\in I_x.
\end{equation}
As a consequence, 
\begin{equation}\label{3.5}
\|u\|_{L^\infty(\Omega_{\rho})} \leq \frac{1}{\gamma} \|u\|_{L^1(\Omega)}  \mbox{ where } \Omega_{\rho}=\{x\in \Omega \colon \mathrm{dist}(x,\partial \Omega)<\rho \}.
\end{equation}
If $\Omega$ is not convex but we assume that $n=2$ and $f\geq 0$, then (\ref{3.5}) also holds for some constants $\rho$ and $\gamma$ depending only on $\Omega$.

\begin{proof}
Use the Method of Moving Plane
\end{proof}
\end{proposition}
\begin{lemma}
If $u$ is solution to 
$$\begin{cases}-\Delta u=f(u) \geq 0  &\mbox{ in } \Omega,\\
 u= 0 & \mbox{ on } \Omega.\\
 \end{cases},$$
 then 
 \begin{equation}\label{uestimate}
 \frac{u}{\delta} \geq c\|f(u)\|_{L_{\delta}^1(\Omega)}   \mbox{ in } \Omega
 \end{equation}
 for some positive constants $c$ and $\rho$ depending only on $\Omega$. Note that 
 $$\|f(u)\|_{L_{\delta}^1(\Omega)} =\int_{\Omega} f(u) \delta dx.$$
 \begin{proof}
 \begin{itemize}
 \item \emph{Step 1}: 
 For any compact set $K\subset \Omega$, we show that 
 \begin{equation}\label{compact}
 u(x)\geq c\int_{\Omega} f(u) \delta   \mbox{ for all } x\in K
 \end{equation}
 where $c$ is a positive constant depending only on $K$ and $\Omega$.
 
 To prove (\ref{compact}), we first define $\rho:=\frac{\mathrm{dist}(K,\partial \Omega)}{2}$ and then take $n$ balls of radius $\rho$ such that $$K\subset B_{\rho}(x_1)\cup\cdots\cup B_{\rho} (x_m)\subset \Omega.$$ This is possible by compactness of $K$.
Let $\xi_1, \xi_2,\cdots, \xi_n$ be the solution of 
$$\begin{cases} -\Delta \xi_i=\chi_{B_{\rho} (x_i)} & \mbox{ in }  \Omega,\\
\xi_i =0   & \mbox{ on } \partial \Omega\\
\end{cases}$$
where $\chi_A$ denotes the characteristic function of $A$. The Hopf boundary lemma implies that there exist $c>0$ such that 
$$\xi_i(x) \geq c\delta(x) \mbox{ for all } x\in \Omega, 1\leq i\leq m.$$
Here and in the rest of the proof, $c$ denotes various constants depending only on $K$ and $\Omega$. 

Let $x\in K$, take a ball $B_{\rho}(x_i)$ containing $x$ , then
$$B_{\rho}(x_i) \subset B_{2\rho }(x)\subset \Omega.$$
\begin{align*}
u(x)\geq & \frac{1}{|B_{2\rho}(x)|} \int_{|B_{2\rho} (x)|} u(x) \mbox{ (By Mean Value Formula)}\\
{}=&c \int_{B_{2\rho} (x)} u(x)\\
{}\geq & c \int_{B_{\rho} (x_i)}u(x)\\
{}=&c\int_{\Omega} u(-\Delta \xi_i) \\
{}=&c\int_{\Omega} f(u) \xi_i\\
{}\geq& c\int_{\Omega} f(u) \delta 
\end{align*}
\item \emph{Step 2:}
Fix a smooth compact set $K\subset \Omega$, by (\ref{compact}), 
$$u\geq c\int_{\Omega} f(u) \delta \mbox { in } K$$
so that it suffices to prove (\ref{compact}) for $x\in\Omega\setminus K.$

Let $w$ be solution of $$\begin{cases} -\Delta w=0 &\mbox{ in } \Omega\setminus K\\
w=0 & \mbox{ on } \partial\Omega\\
w=1 & \mbox{ on } \partial K\\
\end{cases}$$

Then Hopf lemma implies that $w(x)\geq c\delta(x)$ for all $x\in \Omega\setminus K$.

$u$ is a superharmonic and $u((x)\geq c\left(\int_{\Omega} f\delta \right) w(x)\geq c\left(\int_{\Omega} f\delta \right)\delta(x)\mbox{ for } x\in \Omega\setminus K.$
This completes the proof.
 \end{itemize}
 \end{proof}
\end{lemma}
Now we are ready to prove Theorem 2:
\begin{proof}
Assume $f\geq 0$ and $\Omega$ is convex in the case $n\in\{3,4\}$. Let $u_k$ be a sequence of classical positive semistable solutions of (\ref{pde}) converging to $u$ in $L^1(\Omega)$.

For $x\in \Omega$ and $v\colon\Omega\to \mathbb{R}$, define
$$\delta(x):=\mathrm{dist}(x,\partial \Omega) \mbox{ and } \|v\|_{L_{\delta}^1(\Omega}=\|v\delta\|_{L^1(\Omega)}.$$

By Proposition 2, 
\begin{equation}\label{prop2results}
\|u_k\|_{L^{\infty}(\Omega_\rho)}\leq \frac{1}{\gamma} \|u_k\|_{L^1(\Omega)} \to \frac{1}{\gamma}\|u\|_{L^1(\Omega},
\end{equation}
as $k\to\infty$ where $\rho$ and $\gamma$ are positive constants depending only on $\Omega$.

By Lemma 1, we know that 
\begin{equation}\label{lemmaresults}
\frac{u_k}{\delta}\geq c\|f(u_k)\|_{L_{\delta}^1(\Omega)}  \mbox{ in } \Omega
\end{equation}
for some positive $c$ depending only on $\Omega$.

Multiply (\ref{pde}) (with $u$ replaced by $u_k$) by the first Dirichlet eigenfucntion of $-\Delta$ in $\Omega$ and integrate twice by parts:

$$\int -\Delta u_k \phi =\int f(u_k) \phi$$
$$\int -u_k\Delta \phi =\int f(u_k) \phi$$
$$\int -\lambda_1 u_k\phi =\int f(u_k) \phi$$

We deduce that $\|u_k\|_{L_{\delta}^1(\Omega)}$ and $\|f(u_k)\|_{L_{\delta}^1(\Omega)}$ are comparable up to multiplicative constants depending only on $\Omega$.

Similarly, by multiplying (\ref{pde}) by the solution $w$ of 
$$\begin{cases} -\Delta w=1 &\mbox{ in } \Omega,\\
w=0  &\mbox{ on } \partial \Omega .\end{cases}$$

\begin{equation}\label{w}
\int u_k=\int f(u_k) w
\end{equation}
we also deduce that $\|u_k\|_{L^1(\Omega)}$ is comparable to $\|u_k\|_{L_{\delta}^1(\Omega)}$ and $\|f(u_k)\|_{L_{\delta}^1(\Omega)}$.

Recall that $\|u_k\|_{L^1(\Omega)}\to \|u\|_{L^1(\Omega)}>0.$ The RHS of (\ref{lemmaresults}) is bounded below by a positive constant indepedent of $k$.
As a consequence of this lower bound and of (\ref{prop2results}), Proposition 1 gives a uniform $L^\infty(\Omega)$ estimated for all $u_k$. Letting $k\to\infty, u\in L^{\infty}(\Omega)$. Part (i) of Theorem 2 is thus proved.

To prove part(ii), we simply take more precise the constants in (\ref{prop2results}) and (\ref{lemmaresults}). Since we now assume $f\geq c_1>0$,  from (\ref{w}) $u_k\geq c_1 w \geq c_1 c\delta=c_1 c\mathrm{dist} (\cdot,\partial\Omega)$

Finally, multiply (\ref{pde}) ($u_k$ for $u$ ) by the first Dirichlet eigenfunction $-\Delta$ in $\Omega$ and integrate twice by parts. Using the fact taht $f(s)\geq \mu s-c_2$ for all $s, \mu >\lambda_1$,
$$\int -\Delta u_k \phi =\int f(u_k) \phi$$
$$\int -u_k\Delta \phi =\int f(u_k) \phi$$
$$\int -\lambda_1 u_k\phi =\int f(u_k) \phi$$
$$\int -\lambda_1 u_k\phi \geq\int (\mu u_k-c_2) \phi$$
$$\int (\mu -\lambda_1) u_k\phi \leq\int c_2 \phi$$

This shows that $\|u_k\|_{L_{\delta}^{1}(\Omega)} \leq \bar{C}(\Omega,\mu,c_2)$ and also for $\|u_k\|_{L^1(\Omega)}$. By (\ref{prop2results}),
$$\|u\|_{L^\infty(\Omega_{\rho}} \leq \frac{1}{\gamma} \bar{C}(\Omega,\mu,c_2)$$.
Then the result of (ii) of Proposition 2 gives the desired result for Theorem 2.
\end{proof}

The main application of Theorem 1 is the following PDE:
$$\begin{cases} -\Delta u=\lambda g(u) & \mbox{ in } \Omega,\\
u\geq 0 &\mbox{ in } \Omega,\\
u= 0 &\mbox{ on } \partial\Omega,\\
\end{cases}$$
where $\Omega\subset\mathbb{R}^n$ is smooth bounded domain, $n\geq 2, \lambda\geq 0$ and the nonlinearity $g\colon[-,\infty) \to\mathbb{R}$ satisfies 
\begin{equation}\label{g}
g\in C^1, \mbox{ nondecreasing } g(0).0, \mbox {and } \lim_{u\to\infty}\frac{g(u)}{u}=\infty.
\end{equation}
\begin{theorem}
Let $g$ satisfy (\ref{g}) and $\Omega\subset\mathbb{R}^n$ is a smooth bounded domain. Assume that $2\leq n\leq 4$ and $\Omega$ is convex in the case $n\in\{3,4\}$. Let $u^{\star}$ be the extremal solution of the above problem, then $u^{\star}\in L^\infty(\Omega).$
\begin{proof}
\begin{enumerate}
\item \emph{Step 1: } 

We extend $g$ in $C^1$ manner to all of $\mathbb{R}$ with $g$ non-decreasing and $g\geq g(0)/2$ in $\mathbb{R}$. Recall that the extremal solution $u^{\star}$ in the increasing $L^1$ limit as $\lambda\to\lambda^{\star}$, of the minimal solutions $u_\lambda$ of the eigenvalue problem. In addition, for $\lambda<\lambda^{\star},$ $u_\lambda$ is $C^2$-semistable solution of the eigenvalue problem.

\item \emph{Step 2:}

If $g$ is $C^\infty$, we simply apply part (ii) of Theorem 2 with $f=\lambda g$ for $\lambda^{\star}/2<\lambda^{\star}$. Using that $g$ satisfies (\ref{g}), and $f=\lambda g$, we know that $$f(s)\geq \frac{\lambda g(0)}{2}=c_1>0 \mbox{ and } f(s)=\lambda g(s) \geq \mu s-c_2.$$

By Theorem 2, $\|u_\lambda\|_{L^{\infty}(\Omega)}$  are uniformly bounded in $\lambda$. Letting, $\lambda\to\lambda^{\star}$, $u^{\star}\in L^{\infty}(\Omega).$

\item \emph{Step 3:}
If $g\in C^1$ but not $C^\infty$, we use mollifier. Let $\rho_k$ be a $C^\infty$ mollifier with support in $(0,1/k)$ of the form 
$$\rho_k(\beta)=k\rho(k\beta).$$
We replace $g$ by 
$$g_k(s)=\int_{s-1/k}^s g(\tau)\rho_k(s-\tau) d\tau =\int _{0}^1 g(s-\beta/k)\rho (\beta) d\beta.$$

For all $k$, we have $g_k\leq g_{k+1}\leq g$ in $\mathbb{R}$. $g_k$ is $C^\infty$, nondecreasing, and satisfies (\ref{g}).

Since $g(u^\star)\geq g_k(u^\star)$, $u^\star$ is a super -solution to 
$$\begin{cases} -\Delta u=\lambda g_k(u) & \mbox{ in } \Omega,\\
u\geq 0 &\mbox{ in } \Omega,\\
u= 0 &\mbox{ on } \partial\Omega.\\
\end{cases}$$
By the monotone iteration procedure, the extremal parameter for $g_k$, $\lambda_k^\star$ satisfies $\lambda^\star\leq \lambda_{k}^{\star}.$ Hence $u_{\lambda^{\star} -1/k}^k$ is a classical solution to $$-\Delta u=(\lambda^{\star} -1/k)g_k.$$
Thus, we can apply Theorem 1.2 with $f=\lambda g_k$ and $\lambda=\lambda^{\star}-1/k $ to obtain an $L^\infty(\Omega)$ bound for $u_{\lambda^{\star}-1/k}^{k}$ independent of $k$. Note that $u_{\lambda^{\star}-1/k}^{k}\leq u_{\lambda^{\star}-1/(k+1)}^{k}$  and that , since $g_l\leq g_{k+1}\leq g,$ $ u_{\lambda^{\star}-1/(k+1)}^{k}\leq u_{\lambda^{\star}-1/(k+1)}^{k+1}\leq u_{\lambda^{\star}}=u^\star.$ Thus, $u_{\lambda^{\star}-1/k}^k$ increases in $L^1(\Omega)$ towards a solution of  $-\Delta u=\lambda^\star g(u)$ smaller or equal to $u^\star$, and hence identically $u^\star$. From the $L^\infty(\Omega)$ bound for $u_{\lambda^\star-1/k}^k$ independent of $k$, we conclude that $u^\star\in L^\infty(\Omega).$
\end{enumerate}
\end{proof}
\end{theorem}

\section{Open Problems and Future Work}
\begin{enumerate}
\item Theorem 3 for nonconvex domains
\item The boundedness of $u^{\star}$ in the dimensions $5\leq n\leq 9$ 
\end{enumerate}
\end{document}